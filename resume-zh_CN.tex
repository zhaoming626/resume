% !TEX TS-program = xelatex
% !TEX encoding = UTF-8 Unicode
% !Mode:: "TeX:UTF-8"

\documentclass{resume}
\usepackage{zh_CN-Adobefonts_external} % Simplified Chinese Support using external fonts (./fonts/zh_CN-Adobe/)
%\usepackage{zh_CN-Adobefonts_internal} % Simplified Chinese Support using system fonts
\usepackage{linespacing_fix} % disable extra space before next section
\usepackage{cite}

\begin{document}
\pagenumbering{gobble} % suppress displaying page number

\name{赵明}

% {E-mail}{mobilephone}{homepage}
% be careful of _ in emaill address
\contactInfo{zhaoming626@yeah.net}{(+86) 150-6237-6948}{http://mingzhao.online}
% {E-mail}{mobilephone}
% keep the last empty braces!
%\contactInfo{xxx@yuanbin.me}{(+86) 131-221-87xxx}{}
 
\section{\faGraduationCap\  教育背景}
\datedsubsection{\textbf{中国科学技术大学}, 合肥, 安徽}{2014 -- 2017}
\textit{硕士}\ 软件工程
\datedsubsection{\textbf{北京邮电大学}, 北京}{2010 -- 2014}
\textit{学士}\ 通信工程

\section{\faUsers\ 项目经历}
\datedsubsection{\textbf{斯伦贝谢北京研发中心} 北京}{2017 -- 现在}
\role{软件工程师}{组: DevOps组}
Drillplan是自动化钻井设计的一个软件。DevOps组负责整个平台的持续集成和持续部署。
\begin{itemize}
  \item 构建部署脚本的CI测试Pipeline,即时发现Azure和部署脚本的问题。
  \item 根据Contract Test的结果选择适合集成测试的包。
  \item 根据平台功能演化,适当的改进部署脚本。
  \item 参与平台发布周期的维护。
\end{itemize}

\datedsubsection{\textbf{期折优会}}{2016 -- 2017}
\role{软件工程师}{组: iOS客户端开发}
根据提前知道对商品的需求信息可以换取生产者优惠的理解构建平台。
\begin{itemize}
  \item 使用React-Native构建iOS平台程序。
  \item 设计实现展示产品历史价格信息,用户profile的React-Native组件。
  \item 使用Redux设计实现应用的内部状态迁移。
  \item 使用Lint工具确保代码质量。
  \item 参与Code Review和Peer Coding,进行知识迁移。
\end{itemize}

\datedsubsection{\textbf{杭州量子金融有限公司} 浙江}{2015 -- 2016}
\role{实习生}{组:研发}
构建前后端程序销售大数据构建出的金融产品。
\begin{itemize}
  \item 负责分发IOS版客户端的内测和外测,以及IOS版的客户端的上线
  \item 负责Android版的客户端的上线
  \item 对同事写的代码进行审查,确保代码质量
  \item 负责在后端接入中金支付的接口,与中金支付的有关人员沟通绑卡、支付、打款,清算等有关问题
  \item 负责接入短信发送接口,用短信通知用户重要的信息
  \item 使用JQuery, Ionic, Angular, Cordova编写前端,使用Laravel编写后端
\end{itemize}
% Reference Test
%\datedsubsection{\textbf{Paper Title\cite{zaharia2012resilient}}}{May. 2015}
%An xxx optimized for xxx\cite{verma2015large}
%\begin{itemize}
%  \item main contribution
%\end{itemize}

\section{\faCogs\ IT 技能}
% increase linespacing [parsep=0.5ex]
\begin{itemize}[parsep=0.5ex]
  \item 编程语言: (Java = NodeJS) > (Python = PowerShell) > (R)
  \item 平台: Azure, Linux, Angular, React, Git
  \item 开发方法论: 敏捷开发,持续集成,持续发布,测试驱动开发
\end{itemize}

\section{\faInfo\ 其他}
% increase linespacing [parsep=0.5ex]
\begin{itemize}[parsep=0.5ex]
  \item GitHub: https://github.com/zhaoming626
  \item 语言: 英语 - 大学英语六级560,大学英语四级611,CEFR B2
\end{itemize}

%% Reference
%\newpage
%\bibliographystyle{IEEETran}
%\bibliography{mycite}
\end{document}
